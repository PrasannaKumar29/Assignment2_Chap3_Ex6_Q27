\documentclass[journal,12pt,twocolumn]{IEEEtran}
\usepackage{tikz}
\usepackage{amsmath}
\usepackage{amssymb}
\pagestyle{empty}
\usepackage{setspace}
\usepackage{gensymb}
\singlespacing

\usepackage{amsmath}
\usepackage{amsthm}
\begin{document}
\newcommand{\myvec}[1]{\ensuremath{\begin{pmatrix}#1\end{pmatrix}}}
\newcommand{\cmyvec}[1]{\ensuremath{\begin{pmatrix*}[c]#1\end{pmatrix*}}}
\newcommand{\mydet}[1]{\ensuremath{\begin{vmatrix}#1\end{vmatrix}}}
\newcommand{\proj}[2]{\textbf{proj}_{\vec{#1}}\vec{#2}}
\let\StandardTheFigure\thefigure
\let\vec\mathbf

\title{
Assignment - 2
}
\author{ Prasanna Kumar R - SM21MTECH14001}
\maketitle
\newpage
\bigskip
\bibliographystyle{IEEEtran}
\section*{\textbf{Problem}}
\noindent
\textbf{Perpendiculars PL , PM are drawn from P $(h,k)$ on the axes OL,OM. Show that the length of the perpendicular from P on the LM is $hk \sin^2 {\omega} / \sqrt{h^2+k^2+2hk \cos{\omega}}$ and that the equation of the perpendicular is $h(x-h)=k(y-k) $} 
\section*{\textbf{Solution}}
Let PQ be the perpendicular from P on LM, then

Area of $\triangle$ PLM is given as ,
\begin{align}
   \frac{1}{2} {\left\|\mathbf{L-M}\right\|} {\left\|\mathbf{P-Q}\right\|}
\end{align}
Again, taking PM as base and L as vertex,
Area of $\triangle$ PLM is given as ,
\begin{align}
  \frac{1}{2} {\left\|\mathbf{P-M}\right\|} {\left\|\mathbf{P-L}\right\|} \sin{\measuredangle{LPM}}
\end{align}
As $\measuredangle{LPM}= 180^{\circ}- \omega $, by (1) and (2) we get,
\begin{align}
     \frac{1}{2} {\left\|\mathbf{L-M}\right\|} {\left\|\mathbf{P-Q}\right\|}= \frac{1}{2} {\left\|\mathbf{P-M}\right\|} {\left\|\mathbf{P-L}\right\|} \sin{\measuredangle{LPM}}
\end{align}
Here,
\begin{align*}
{\left\|\mathbf{L-M}\right\|} &= \sin{\omega} \sqrt{h^2+k^2+2hk \cos{\omega}} \\[6pt]
{\left\|\mathbf{P-M}\right\|} &= k \sin{\omega} \\[6pt]
{\left\|\mathbf{P-L}\right\|} &= h \sin{\omega} \\[6pt]
\end{align*}
\begin{align}
\begin{split}
    (\sin{\omega} \sqrt{h^2+k^2+2hk \cos{\omega}}) {\left\|\mathbf{P-Q}\right\|} &= \\  k \sin{\omega} \ h \sin{\omega} \  \sin{(180^{\circ}- \omega)} \\[6pt]
    {\left\|\mathbf{P-Q}\right\|} &= \\ \frac{hk \sin^2 {\omega}}{\sqrt{h^2+k^2+2hk \cos{\omega}}}
\end{split}
\end{align}
Equation of LM is,
\begin{align}
\begin{split}
 \frac{x}{\vec{OM}}+ \frac{y}{\vec{OL}} = 1   \\
 \frac{x}{h+ k \cos{\omega}}+ \frac{y}{k+ h\cos{\omega} } = 1
\end{split}
\end{align}
Equation of any line through $(h,k)$ will be, 
\begin{align}
    \begin{split}
        y-k= m (x-h)
    \end{split}
\end{align}
As lines LM and PQ are perpendicular, 
\begin{align}
    \begin{split}
        1+ \left(m -\frac{k+ h\cos{\omega}}{h+ k \cos{\omega}}\right)\cos{\omega}- m \left(\frac{k+ h\cos{\omega}}{h+ k \cos{\omega}}\right) =0 
    \end{split}
\end{align}
Solving the above equation, $$m= \frac{h}{k}$$
Substituting $m$ in (7), we get
\begin{align}
\begin{split}
(y-k)= \frac{h}{k} (x-h) \\
\Rightarrow h(x-h)= k(y-k)
\end{split}
\end{align}

\begin{figure}[htp]
    \centering
    \includegraphics[width=8cm]{123.jpg}
    \caption{Diagram}
    \label{fig:galaxy}
\end{figure}
\end{document}